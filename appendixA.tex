\chapter{Implementación de GPT} \label{appendixA}
GPT es un modelo sencillo y puede implementarse elegantemente en pocos cientos de líneas de código. Se presenta aquí una implementación funcional de los bloques fundamentales, que hace uso de la biblioteca \textit{pytorch}.

La implementación de la atención multicabezal puede hacerse de forma muy similar a la explicada anteriormente (\cref{code:attn}), añadiendo \textit{abandono} en este caso. A fin de poder utilizar \textit{teacher forcing} de forma eficiente durante el entrenamiento, se \textit{enmascaran} los símbolos a la derecha del símbolo actual en la secuencia de entrada reemplazando esas posiciones con \( -\infty \), de forma que la activación de salida asociada a esa entrada sea \( 0 \).

\begin{code}
\begin{minted}[frame=single,framesep=5pt,breaklines,fontsize={\fontsize{9.5}{10.5}\selectfont}]{python}
class MultiheadAttention(Module):
  def __init__(self, d_embedding, n_head, dropout, block_size):
    super().__init__()
    self.d_embedding = d_embedding
    self.n_head = n_head

    self.c_attn = nn.Linear(config.d_embedding, 3 * d_embedding)
    self.c_proj = nn.Linear(config.d_embedding, d_embedding)
    self.attn_dropout = nn.Dropout(dropout)
    self.resid_dropout = nn.Dropout(dropout)
    self.dropout = dropout
    self.register_buffer("attn_mask", torch.tril(torch.ones(block_size, block_size)).view(1, 1, block_size, block_size))

  def forward(self, x):
    B, N, D = x.size()
    q, k, v = self.c_attn(x).split(self.d_embedding, dim=2)
    q = q.view(B, N, self.n_head, D // self.n_head).transpose(1, 2)
    k = k.view(B, N, self.n_head, D // self.n_head).transpose(1, 2)
    v = v.view(B, N, self.n_head, D // self.n_head).transpose(1, 2)

    attn = (q @ k.transpose(-2, -1)) * (1.0 / math.sqrt(k.size(-1)))
    attn = softmax(attn.masked_fill(self.attn_mask[:,:,:T,:T] == 0, float('-inf')))
    attn = self.attn_dropout(attn)
    y = attn @ v
    y = y.transpose(1, 2).contiguous().view(B, T, C)
    y = self.resid_dropout(self.c_proj(y))
    return y
\end{minted}
\caption{Implementación en \textit{Pytorch} de la atención multicabezal}
\label{code:attn}
\end{code}

La dimensión de la representación de cada símbolo (\texttt{d\_embedding}), el número de cabezales de atención (\texttt{n\_head}), el factor de abandono (\texttt{dropout}) y la longitud de contexto (\texttt{block\_size}) se mantienen como parámetros, a fin de poder reutilizar la implementación para definir distintas versiones de GPT. 

\begin{code}
\begin{minted}[frame=single,framesep=5pt,breaklines,fontsize={\fontsize{9.5}{10.5}\selectfont}]{python}
def MLP(d_embedding, dropout):
  return nn.Sequential(OrderedDict([
    ("c_fc", nn.Linear(d_embedding, 4 * d_embedding)),
    ("act", nn.GELU()),
    ("c_proj", nn.Linear(4 * d_embedding, d_embedding)),
    ("dropout", nn.Dropout(dropout))
  ]))

class Block(nn.Module):
  def __init__(self, d_embedding, n_head, dropout, block_size):
    super().__init__()
    self.ln_1 = LayerNorm(d_embedding)
    self.attn = MultiheadAttention(d_embedding, n_head, dropout, block_size)
    self.ln_2 = LayerNorm(d_embedding)
    self.mlp = MLP(d_embedding, dropout)

  def forward(self, x):
    x = x + self.attn(self.ln_1(x))
    x = x + self.mlp(self.ln_2(x))
    return x
\end{minted}
\caption{Implementación en \textit{Pytorch} del bloque decodificador del transformer}
\label{code:decoder}
\end{code}

\begin{code}
\begin{minted}[frame=single,framesep=5pt,breaklines,fontsize={\fontsize{9.5}{10.5}\selectfont}]{python}
class GPT(Module):
  def __init__(self, d_embedding, n_layer, n_embd, vocab_size, block_size, dropout):
    self.transformer = nn.ModuleDict(dict(
      wte = Embedding(vocab_size, d_embedding),
      wpe = Embedding(block_size, d_embedding),
      drop = Dropout,
      h = ModuleList([Block(d_embedding, n_head, dropout, block_size) 
        for _ in range(n_layer)]),
      ln_f = LayerNorm(n_embd)))
    self.lm_head = nn.Linear(config.d_embedding, config.vocab_size, bias = False)

  def forward(self, x):
    _, t = x.size()
    pos = torch.arange(0, t).unsqueeze(0)
    tok_emb = self.transformer.wte(idx) # representación de los símbolos
    pos_emb = self.transformer.wpe(pos) # representación de las posiciones

    x = self.transformer.drop(tok_emb + pos_emb)
    for block in self.transformer.h:
      x = block(x)
    x = self.transformer.ln_f(x)
    return self.lm_head(x)
\end{minted}
\caption{Implementación en \textit{Pytorch} de la arquitectura GPT}
\label{code:gpt}
\end{code}

El bloque decodificador del transformer, la base de GPT, se presenta en el \cref{code:decoder}. A partir de estos componentes, se pueden definir distintas arquitecturas inspiradas en GPT (\cref{code:gpt}), variando el número de bloques o la dimensión de las representaciones internas.
